Computer vision aims to extract meaningful information from images, in order to process this information as objects for specific purposes, such as environment recognition and motion detection. One of the main challenges of this area is the 3D reconstruction from images, which is very useful as a tool for other areas of knowledge. This work investigate the feasibility of using HDR (High Dynamic Range) images for increasing the quality of 3D objects generated from images. For that, it was made a literature review on generating HDR images from LDR (Low Dynamic Range) images, and about obtaining point cloud from images. The work was made in two stages. On the first stage methods for generating HDR were studied, implemented and compared. On the second stage were explored possibilities for using HDR images on point cloud generation. The results obtained on those experiments were compared with point clouds generated using LDR images, what showed the potential of the method proposed in relation to conventional methods.

\textbf{Keywords:} HDR, LDR, point cloud, multiple exposures, feature points, response function, 3D reconstruction. 
\chapter{Considerações Finais} \label{conclusao}

Este trabalho apresentou métodos para geração de imagens HDR a partir de um conjunto de imagens LDR, e abordou a viabilidade do uso de imagens HDR para a obtenção de nuvem de pontos 3D. O trabalho foi dividido em duas etapas. A primeira etapa visou definir um método de geração de imagens HDR viável, para ser utilizado posteriormente na obtenção de nuvem de pontos a partir de imagens HDR. A segunda etapa consistiu na verificação de procedimentos para maximizar a quantidade de pontos obtidos em relação ao método convencional de obtenção de nuvem de pontos. 

Dentre os métodos de geração de imagens HDR foram implementados e testados os métodos propostos por Mann e Picard~\cite{mann} e Robertson~\etal~\cite{robertson} e apenas testado o método proposto por  Sen~\etal~\cite{hdrMovimento}. O método escolhido para ser utilizado na segunda etapa, foi o método proposto por Sen~\etal~\cite{hdrMovimento}, por possuir maior robustez quanto ao movimento da câmera.

Na segunda etapa do trabalho, foram comparados três métodos para a obtenção de nuvem de pontos a partir de imagens. O primeiro método consiste na forma convencional, onde o usuário se utiliza de sua sensibilidade para selecionar imagens LDR que considere bem expostas e que possuam menor perda de informação. O segundo método consiste em utilizar imagens HDR \textit{tonemapped} para a obtenção dos pontos. E o terceiro consiste no uso de imagens HDR \textit{tonemapped} processadas por um filtro de realce de contornos para obtenção da nuvem de pontos.

Os testes foram feitos com duas bases de imagens. A primeira base consiste num ambiente controlado, com pouco movimento da câmera durante as capturas. A segunda base consiste numa situação de usuário comum, onde as imagens são capturadas sem uso de equipamentos profissionais, como tripé. Os resultados obtidos indicam que o uso de imagens HDR aumenta a quantidade de pontos obtidos na nuvem. Este aumento foi bastante significativo para o ambiente controlado, possuindo aumento no número de pontos obtidos de até $800\%$ em relação ao método convencional.

Como sugestões de trabalhos futuros pode-se fazer análise qualitativa dos pontos obtidos com os métodos deste trabalho, verificando a precisão em relação a um modelo de referência e aumentando a base de testes. Também pode ser verificada a aplicação do realce de contornos, antes do \textit{tone mapping}, e seus efeitos sobre a nuvem de pontos gerada. Outra possibilidade de trabalho futuro é o estudo de técnicas de \textit{tone mapping} que aumentem a quantidade e/ou qualidade dos pontos da nuvem gerada.

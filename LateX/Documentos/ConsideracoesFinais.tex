\chapter{Considerações Finais} \label{conclusao}

Com base nos resultados conclui-se que o uso de imagens HDR obtidas a partir de um conjunto de imagens LDR com diferentes tempos de exposição não se mostra uma alternativa viável para a geração de nuvens de pontos 3D, tendo em vista que as imagens LDR não são capturadas com uso de ferramentas profissionais como tripé.

Os resultados mostraram que as imagens HDR \textit{tonemapped}, que deveriam possuir mais informações do ambiente em relação a uma LDR comum, geraram nuvem de pontos de qualidade inferior à nuvem de pontos obtida utilizando imagens LDR convencionais. O fator que se mostrou limitante para o processo foi a inerente movimentação da câmera entre a captura das imagens LDR, que compuseram as imagens HDR. O fato das imagens não serem alinhadas acabou por gerar artefatos nas imagens HDR o que dificultou o processo de obtenção dos pontos 3D ao invés de melhorá-lo.

Como trabalhos futuros pode-se analisar o processo abordado neste trabalho utilizando-se de ferramentas profissionais que possibilitem a captura de imagens em diferentes tempos de exposição sem que haja movimento entre duas capturas.




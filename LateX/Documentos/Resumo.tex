Este documento mostra a pesquisa e implementação de métodos de geração de imagens HDR a partir de um conjunto de imagens LDR. A literatura apresenta vários métodos para a geração de imagens com alta faixa dinâmica como os métodos de Mann e Picard que foi um dos primeiros métodos propostos, o método de Robertson~\etal, que trata do ruído da câmera sob a ótica estatistica e o método de Sen~\etal, que trata o problema com o enfoque em robustez quanto ao movimento. O principal objetivo deste trabalho é definir um método viável para ser utilizado na geração de imagens HDR que servirão de entrada para um posterior trabalho de obtenção de nuvem de pontos a partir de imagens HDR. Os resultados obtidos foram satisfatórios para todos os métodos abordados, sendo então escolhido o método mais robusto quanto ao movimento para os próximos passos do projeto.
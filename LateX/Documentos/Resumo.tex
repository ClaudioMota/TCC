A visão computacional é uma área que possui como objetivo extrair informações significativas a partir de imagens, e assim possibilitar o processamento dessas informações como objetos para fins específicos como o reconhecimento de um ambiente e detecção de movimento. Um dos principais desafios desta área é a reconstrução de objetos 3D a partir de imagens, desafio este que é de grande utilidade, como ferramenta, para outras áreas do conhecimento. Este trabalho investiga a viabilidade do uso de imagens HDR (do inglês, \textit{High Dynamic Range}) para aumentar a qualidade de objetos 3D gerados a partir de imagens. Para isso foi feita a revisão da literatura sobre geração de imagens HDR a partir de imagens LDR (do inglês, \textit{Low Dynamic Range}), e sobre a obtenção de nuvem de pontos a partir de imagens. O trabalho foi dividido em duas etapas. Na primeira etapa foram estudados, implementados e comparados métodos de geração de imagens HDR. A segunda etapa do trabalho consistiu na exploração das possibilidades de uso de imagens HDR na geração de uma nuvem de pontos. Os resultados obtidos nestes experimentos foram comparados com nuvens de pontos geradas a partir de imagens LDR, demostrando assim o potencial da abordagem proposta em relação aos métodos convencionais.

\textbf{Palavras-chave:} HDR, LDR, nuvem de pontos, múltiplas exposições, pontos de inte-resse, função resposta, reconstrução 3D. 
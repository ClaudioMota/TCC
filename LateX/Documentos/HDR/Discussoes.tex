\section{Discussões} \label{discussoes}

Com os resultados alcançados, notou-se que todos os métodos atingiram os resultados esperados (geração de imagens HDR), cada qual com suas vantagens e desvantagens. Nesta seção serão abordadas as questões referentes a cada método de geração de imagens HDR abordado neste documento.

\subsection{Método Mann} \label{discussaoMann}
\subsubsection{Vantagens}
\begin{itemize}
	\item Possui a inferência da função resposta mais rápida.
	\item Em alguns casos mostrou resultado mais consistente que o método de Robertson \etal \cite{robertson}.
\end{itemize}
\subsubsection{Desvantagens}
\begin{itemize}
	\item As imagens de entrada não devem apresentar movimento entre suas capturas.
	\item A inferência da função resposta leva em conta apenas alguns pixels e não todos.
	\item A inferência da função resposta leva em conta apenas 2 imagens.
	\item A falta de especificação da escala na inferência da função resposta leva, em alguns casos, à incoerência na união dos canais de cores.
\end{itemize}


\subsection{Método Robertson} \label{discussaoRobertson}
\subsubsection{Vantagens}
\begin{itemize}
	\item A inferência da função resposta leva em conta todas as imagens.
	\item A inferência da função resposta leva em conta todos os pixels da imagem.
\end{itemize}
\subsubsection{Desvantagens}
\begin{itemize}
	\item Na implementação feita algumas imagens HDR apresentaram ruído.
	\item As imagens de entrada não devem apresentar movimento entre suas capturas.
\end{itemize}

\subsection{Método Sen} \label{discussaoSen}
\subsubsection{Vantagens}
\begin{itemize}
	\item Robustez quanto ao movimento nas imagens.
\end{itemize}
\subsubsection{Desvantagens}
\begin{itemize}
	\item Necessita de função resposta.
	\item Método mais lento de obtenção de imagens HDR.
	\item Não suporta imagens em alta resolução.
\end{itemize}

Tanto o método proposto por Mann e Picard~\cite{mann}, quanto o proposto Robertson~\etal~\cite{robertson}, supõem que a câmera estará totalmente estática entre a captura das imagens. Para que isso seja verdade, há a necessidade de equipamentos sofisticados como tripé e software de captura por computador, além de um ambiente estático. 

Usuários comuns de câmeras digitais não costumam possuir equipamento que permitam a fixação da câmera no momento da aquisição das imagens, e o ambiente a ser registrado não costuma ser controlado. Assim, o método mais apropriado se mostra o proposto por Sen~\etal~\cite{hdrMovimento}, por se tratar de um método mais robusto quanto ao movimento. Sendo assim, este será o método utilizado para a próxima etapa deste trabalho para a obtenção de nuvem de pontos a partir de imagens HDR.

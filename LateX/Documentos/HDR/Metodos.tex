\section{Métodos HDR Selecionados} \label{metodos}
Nesta seção serão abordados os métodos de geração de imagens HDR a partir de imagens LDR escolhidos para este trabalho. Cada subseção tratará um método, mostrando aspectos teóricos, implementação e resultados obtidos, assim como discussões sobre tópicos de interesse. Nas seções que seguem serão abordados e comparados os métodos de geração de imagens HDR dos seguintes autores:

\begin{itemize}
\item Mann e Picard: por se tratar de um método clássico, frequentemente citado em trabalhos da área e que pode ser usado como base para comparação. Além disso, trata-se de um método simples que aborda apenas a mudança no tempo de exposição das imagens LDR para geração de imagens HDR, e se mostra suficiente para o objetivo final deste trabalho.
\item Robertson \etal: por se tratar de um método mais elaborado que, além de abordar a mudança no tempo de exposição, leva em consideração aspectos de erro relativos a ruídos na aquisição das imagens LDR. 
\item Sen \etal: por se tratar de um método robusto quanto ao movimento da câmera. Característica essa bastante importante para aplicações não profissionais, pois manter a câmera totalmente estática entre a captura das imagens é um fator limitante.
\end{itemize}


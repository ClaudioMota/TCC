\section{Proposta para Melhoria na Aquisição da Nuvem de Pontos} \label{pontosProposta}

Além do uso de imagens HDR tonemapped, outra proposta deste trabalho para melhorar a qualidade e quantidade de elementos da nuvem de pontos gerada, se baseia no processamento de realce de bordas dessas imagens. A hipótese é que com bordas mais acentuadas os algorítmos de identificação de pontos de interesse identificarão mais pontos, com maior qualidade.

Para isso foi proposto o uso do filtro de Laplace, que se trata de um método conhecido em processamento de imagens para realce de bordas. Este método se baseia no conceito da soma das derivadas parciais de segunda ordem \cite{gonzalez} para realçar as bordas da imagem. A Equação \ref{eqPontosLaplace} mostra operador de Laplace: \cite{gonzalez}

\begin{align} \label{eqPontosLaplace}
	\nabla f^2 = \frac{\partial^2f}{\partial x^2} + \frac{\partial^2f}{\partial y^2}
\end{align}

Para o processamento digital de imagens essa equação é transformada para a sua forma discreta como mostrado na Equação \ref{eqPontosLaplaceDisc}. \cite{gonzalez}

\begin{align} \label{eqPontosLaplaceDisc}
	\nabla f^2 = -4f(x,y) + f(x+1,y) + f(x-1,y) + f(x,y+1) + f(x,y-1)
\end{align}

Segundo Gonzalez \cite{gonzalez}, a Equação \ref{eqPontosLaplaceDisc} resulta numa máscara que leva em consideração as variações de cores dos pixels vizinhos que formam ângulos múltiplos de $90º$ em relação ao pixel central. Gonzalez \cite{gonzalez} ainda acrescenta as diagonais podem ser adicionadas a máscara resultando na matriz mostrada mostrada na Figura \ref{figPontosLaplaceMat}.

\begin{figure}[H]
  \centering
  \includegraphics[height=3cm]{matrizLaplace}
  \caption{Filtro de Laplace.}
  \label{figPontosLaplaceMat}
\end{figure}

Uma vez com as bordas da imagem sendo extraida com o filtro de Laplace. É feito a soma da imagem inicial com as bordas extraídas para que seja obtido o realce das bordas. Neste processo será utilizado um fator $\alpha$, $0 < \alpha < 1$, para regular a intensidade de realce das bordas. A Equação \ref{eqPontosLaplaceSoma} ilustra o processo de realce das bordas.

\begin{align} \label{eqPontosLaplaceSoma}
	Imagem_{final} = Imagem_{inicial} + \alpha f(Imagem_{inicial})
\end{align}
Onde
\begin{itemize}
\item $f$ é o filtro de Laplace.
\end{itemize}

  Sendo assim, o processo geral proposto para obtenção da nuvem de pontos a partir de imagens HDR segue as seguintes etapas:

\begin{itemize}
\item Geração das imagens HDR a partir de conjuntos de imagens LDR.
\item \textit{Tone mapping} das imagens HDR geradas.
\item Realce das bordas das imagens \textit{tonemapped}.
\item Obtenção da nuvem de pontos a partir das imagens processadas.
\end{itemize}
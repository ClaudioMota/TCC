\chapter{Introdução} \label{introducao}
A visão computacional é uma área que possui como objetivo extrair informações significativas a partir de imagens, e assim possibilitar o processamento dessas informações como objetos para fins específicos como o reconhecimento de um ambiente, detecção de movimento entre outros \cite{ballard&brown}. Um dos principais desafios desta área é a reconstrução de objetos 3D a partir de imagens, desafio este que é de grande utilidade, como ferramenta, para outras áreas do conhecimento. Isso é mostrado nos trabalhos de Matsushigue \etal~\cite{matsushigue}, que abordou o uso de reconstrução de tomografias 3D para melhoria de diagnostico de fraturas na região do úmero, e como o trabalho de Andrade \etal~\cite{beatriz}, que usou métodos de reconstrução 3D para a preservação digital de obras de arte.


Outro exemplo de aplicação da reconstrução 3D a partir de imagens foi apresentada no método proposto por Argawal \etal~\cite{agarwal}. O método proposto aborda a reconstrução 3D sobre a escala de cidades utilizando-se de fotos disponíveis na Internet, que por sua vez, foram capturadas por pessoas aleatórias. Para isto foi usada como objeto de estudo a cidade de Roma que, por se tratar de uma cidade turística, possui milhares fotos disponíveis na Internet. Como resultado do trabalho, foi feita a reconstrução 3D do monumento do Coliseu de Roma.


De maneira geral, grande parte das técnicas de reconstrução 3D funcionam com base no mapeamento de pontos singulares existentes em diferentes imagens, denominados pontos de interesse. Com o uso de conjuntos de pontos de interesse comuns entre as imagens é possível estimar a posição deles em relação a um sistema de coordenadas tridimensional, gerando assim uma nuvem de pontos 3D.


Com base na revisão de literatura feita neste trabalho sobre obtenção de nuvem de pontos 3D a partir de imagens, evidenciou-se que grande parte dos métodos existentes utiliza-se de imagens com baixa faixa dinâmica (em inglês, Low Dynamic Range - LDR), que possuem uma resolução limitada de cores. Por outro lado existem também as imagens com alta faixa dinâmica (em inglês, High Dynamic Range - HDR), que possuem maior resolução de cores.

Sob o ponto de vista de acessibilidade, o uso de imagens LDR é explicado pelo fato de que câmeras com sesores capazes de capturar imagens HDR serem muito caras e assim pouco acessíveis. Porém sob o ponto de vista de qualidade, as imagens HDR possuem maior quantidade de informações do cenário em relação a uma imagem LDR. Um exemplo disso, seria um cenário de uma sala pouco iluminada, que possui uma janela ao fundo a qual mostra o céu de um dia ensolarado. Ao capturar uma imagem deste ambiente com uma câmera LDR, o fotógrafo terá que ajustar as configurações da câmera, para escolher qual parte da cena será melhor registrada (os móveis da sala ou o céu ensolarado). Já para uma câmera HDR todas as partes da imagem seriam propriamente registradas, uma vez que esta possui uma maior faixa de representação de cores.


Trabalhos publicados na literatura, propoem métodos para união de um conjunto imagens LDR em uma imagem HDR. O uso destas técnicas se mostra uma alternativa economicamente viável para obter imagens de boa qualidade, com alta faixa dinâmica.


Assim como no trabalho feito por Kontogianni~\etal~\cite{hdr3d}, o trabalho descrito neste documento possui como hipótese a ideia que um conjunto de imagens HDR têm o potencial de identificar mais pontos de interesse que um conjunto de imagens LDR, tendo em vista que o primeiro possui mais informação do ambiente que o segundo. Neste contexto, este trabalho visa verificar a viabilidade de implementação de um método para obtenção de nuvem de pontos a partir de imagens HDR. Espera-se que com isso seja possível extrair mais pontos de interesse e reconstruir nuvens de pontos com maior resolução de maneira acessível. Para que este processo possa ser realizado com um baixo custo em relação ao equipamento utilizado, se faz necessário o uso das técnicas de união de imagens LDR para geração de imagens HDR.

\section{Objetivo} \label{introducaoObjetivo}

O objetivo deste trabalho é verificar se o uso de imagens HDR, obtidas utilizando conjuntos de imagens LDR, gera nuvens de pontos mais densas em relação às nuvens de pontos geradas a partir de imagens LDR.

\section{Metodologia} \label{introducaoMetodo}

Para que seja possível alcançar o objetivo, este trabalho é dividido em duas etapas:

\begin{itemize}
\item Geração de imagem HDR:

Nesta etapa métodos de geração de imagens HDR são estudados, implementados e comparados para que sejam determinados os métodos que serão utilizados na próxima etapa.
\item Geração de nuvem de pontos:

Nesta etapa são utilizados métodos para geração de nuvem de pontos a partir de imagens, para que assim sejam realizadas as comparações inicialmente propostas.
\end{itemize}
	